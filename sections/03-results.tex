\section{RESULTADOS E DISCUSSÕES}

\subsection{Como problemas éticos impactam no desenvolvimento de tecnologias}

É imprescindível que a sociedade avance cada vez mais tecnologicamente. Porem esses avanços  podem trazer conflitos para a sociedade

\subsubsection{Tecnologias voltadas para o âmbito organizacional}

Com o avanço tecnológico as empresas certamente vão querer utilizar essas tecnologias para aumentar sua produtividade e melhorar sua produção. Com isso surgem investimentos na area de automação industrial \cite{Industri87:online}, mas também no quesito de gerenciamento de pessoas. Assim surgem novas tecnologias que visam aumentar essa produtividade.

Porém, como mostrado em \cite{Telkamp2022} o uso de inteligencia artificial para monitorar os empregados gera questionamentos éticos acerca dessas praticas gerarem abuso. A criação de sistemas que monitoram empregados ja foi aplicado pela \textit{Amazon}, mas não faltam relatos de que essas praticas são abusivas para com os empregados \cite{jef:online} \cite{Amazon:online}. 

\subsubsection{Tecnologias com capacidade de previsão}

No estudo realizado pela Universidade de Chicago é utilizado o machine learning para prever, através de areas de incidência, crimes futuros \cite{rotaru2021precise}. Embora realizar a previsão de crimes seja bastante positivo para a sociedade outros problemas podem ser gerados. Para melhorar o sistema de previsão o governo pode coletar os seus dados pessoais ferindo sua privacidade.


\subsection{Maquinas que tomam decisões éticas}

Devido ao uso de inteligencia artificial e a crescente automação das maquinas, os dispositivos são cada vez mais responsáveis por uma maior tomada de decisões. Porem essa tomadas de decisões podem carregar consigo um forte aspecto moral. Como em dilemas com carros autonomos, ou em triagem de hospitais. Assim as maquinas precisam ter uma base moral nelas para realizar a tomada de decisões.

